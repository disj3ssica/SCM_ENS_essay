\documentclass[12pt,a4paper,draft]{article}
\usepackage{xcolor}

\begin{document}

\begin{titlepage}
\title{SCM application to environmental analysis}
\author{Jessica Cremonese}
\date{March 2023}
\maketitle
\end{titlepage}

\tableofcontents

\newpage
\begin{abstract}
    \textcolor{red}{Lorem ipsum}
\end{abstract}
\newpage


\section{Introduction}
\textcolor{red}{Lorem ipsum}




\section{Synthetic Control Methods}
Synthetic Control Methods have been originally proposed in Abadie and Gardeazabal (2003) and by Abadie et al. (2010) to estimate the effects of aggregate interventions.
The key idea behind the method is that when the units of analysis are a few aggregate entities, a "synthetic control" that combines them can be a better counterfactual.
Such a control is computed as a weighted average of all potential comparison units that best resemble the treated units.


\section{Literature review and general applications in macro}
\textcolor{red}{Lorem ipsum}




\section{Potential applications to Fleurbeay project}
\textcolor{red}{Lorem ipsum}




\section{Data sourcing and explanations}
\textcolor{red}{Lorem ipsum}




\section{Conclusion}
\textcolor{red}{Lorem ipsum}



\section{References}
\begin{itemize}
    \item Abadie, Alberto, Alexis Diamond, and Jens Hainmueller. "Synthetic control methods for comparative case studies: Estimating the effect of California’s tobacco control program." Journal of the American statistical Association 105.490 (2010): 493-505.
    \item Abadie, Alberto, and Javier Gardeazabal. "The economic costs of conflict: A case study of the Basque Country." American economic review 93.1 (2003): 113-132.
    \item Abadie, Alberto. "Using synthetic controls: Feasibility, data requirements, and methodological aspects." Journal of Economic Literature 59.2 (2021): 391-425.
\end{itemize}




\end{document}