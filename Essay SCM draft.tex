\documentclass[12pt,a4paper,draft]{article}
\usepackage{xcolor}

\begin{document}

\begin{titlepage}
\title{SCM application to environmental analysis}
\author{Jessica Cremonese}
\date{March 2023}
\maketitle
\end{titlepage}

\tableofcontents

\newpage
\begin{abstract}
    \textcolor{red}{Lorem ipsum}
\end{abstract}
\newpage


\section{Introduction}
\textcolor{red}{Lorem ipsum}




\section{Synthetic Control Methods}
Synthetic Control Methods (SCM) have been originally proposed in Abadie and Gardeazabal 
(2003) and by Abadie et al. (2010) to estimate the effects of aggregate interventions.
The key idea behind the method is that, when units are a few aggregate entities, 
a better counterfactual than using any single unit can be derived by computing a 
combination of the untreated units that closely resembles the treated one, i.e. 
a ``synthetic control". The selection of the ``donor units" is formalized with a data
driven procedure.
Although the method was originally intended for samples with few units, it has been 
successfully applied in contexts with large samples, for instance in Acemoglu et 
al. (2016).
Such a synthetic control unit is computed as a weighted average of all potential 
comparison units that best resemble the treated units. In this section, I will 
introduce the method and explore feasibility, data requirements and methodological 
issues. 
The main references are Abadie and Gardeazabal (2003) and Abadie, Diamond and 
Hainmueller (2010), which introduced the method in the literature, and Abadie 
(2021), which provides a useful guide to the application of SCM.

\subsection{Setting of SCM}
Suppose to have data for $j=1,..,J+1$ units, and suppose that unit $j=1$ is the 
treated unit. The ``donor pool" of untreated units which will contribute to the 
construction of a synthetic control for unit $j=1$ is then constituted by the 
remaining $j=2,...,J+1$ units.
Assume that data covers $T$ periods, with periods up to $T_0$ being the 
pre-intervention observations.

For each unit $j$ at time $t$ data is available for the outcome of interest 
$Y_{jt}$, and for a number $k$ of predictors $X_{1j}, ..., X_{kj}$. Define the 
$k \times 1$ vectors $\mathbf{X}_1, ..., \mathbf{X}_{J+1}$ which contain values
of the predictors for units $j=1,...,J+1$. Define the $k \times J$ matrix 
$\mathbf{X}_0={\mathbf{X}_2,..., \mathbf{X}_{J+1}}$ which collects values of the predictors 
for the untreated units.
For each unit $j$, define the potential outcome without treatment as $Y_{jt}^N$.
For the trated unit $j=1$, define the potential response under the treatment as 
$Y_{jt}^I$ in the post treatment period $t>T_0$. The effect of the intervetion 
for the affected unit $j=1$ for $t>T_0$ is: 
$$\tau_{1t}=Y_{1t}^I-Y_{1t}^N$$.

For the treated unit, $Y_{1t}^I$ is observed so that  $Y_{1t}=Y_{1t}^I$, but $Y_{jt}^N$ is not. 
SCM provides a way to estimate $Y_{jt}^N$ for $t>T_0$, that is, how the outcome 
of interest would have been in the absence of treatment. Notice that $\tau_{1t}$
is allowed to change over time.

\subsection{Estimating}








\section{Literature review and general applications in macro}
\textcolor{red}{Lorem ipsum}




\section{Potential applications to Fleurbeay project}
\textcolor{red}{Lorem ipsum}




\section{Data sourcing and explanations}
\textcolor{red}{Lorem ipsum}




\section{Conclusion}
\textcolor{red}{Lorem ipsum}



\section{References}
\begin{itemize}
    \item Abadie, Alberto, Alexis Diamond, and Jens Hainmueller. ``Synthetic control methods for comparative case studies: Estimating the effect of California's tobacco control program." Journal of the American statistical Association 105.490 (2010): 493-505.
    \item Abadie, Alberto, and Javier Gardeazabal. ``The economic costs of conflict: A case study of the Basque Country." American economic review 93.1 (2003): 113-132.
    \item Abadie, Alberto. ``Using synthetic controls: Feasibility, data requirements, and methodological aspects." Journal of Economic Literature 59.2 (2021): 391-425.
    \item Acemoglu, Daron, et al. ``The value of connections in turbulent times: Evidence from the United States." Journal of Financial Economics 121.2 (2016): 368-391.
\end{itemize}




\end{document}